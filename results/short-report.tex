
\documentclass{article}
\usepackage[utf8]{inputenc}
\usepackage{fullpage}
\usepackage{listings}
\usepackage{float}

% math
\usepackage{mathtools}
\usepackage{amsthm}

% math fonts
\usepackage{amssymb}
\usepackage{dsfont}

\usepackage[dvipsnames]{xcolor}
\usepackage{tikz,pgfplots}
\usepackage{pgfplotstable}

\setlength{\tabcolsep}{2pt}

\pgfplotsset{compat=newest,compat/show suggested version=false}



\setlength\parindent{0pt}

\begin{document}
We tested\footnote{Executed on the BwUniCluster 2.0 on instances with 64GB of RAM on two Cores of an Intel Xeon Gold 6230. For each method we set a time limit of 10 hours to solve all discretizations of a problem. Furthermore, we limited the maximum RAM usage to 64GB. } our methods against a classical time stepping approach. To get a fair comparison, we have implemented a Crank-Nicolson scheme which utilizes the sparseness of the space matrices via an CG method. 

For sufficient smooth examples, we were not able to beat the performance of the time stepping method due to the quadratic $L_2$ convergence of this method - in contrast to our observed linear convergence in the space-time setting. For less smooth examples, the numerical results are more diverse. We want to report some results in 2D and 3D, as the 1D case is mostly of academic nature.

\scriptsize
\section*{2D}
\subsection*{Example 1 ($u \in C^\infty$)}
\begin{tikzpicture}[xscale=.8,yscale=.8]
	\begin{semilogyaxis}[title={2D Example 1: $L_2$ convergence},
			xlabel={Number of refinements},
			ylabel={$L_2$ error},
			grid=major,
			ymax=1e2,
			legend style={at={(1,1)},xshift=2cm,yshift=-0.2cm,anchor=north east,nodes=right}
			]	
	% CG Optimal
	\addplot+ [blue,very thick,mark=x] table [x=refinements, y=errorCGopt] {data/2Dexample1-CG-opt-exact-1-maxIt-1000-tolerance-1e-10-toleranceRes-0.01.txt};
	\addlegendentry{CG Optimal}
	\addplot+ [blue,dotted,very thick,mark=x] table [x=refinements, y=errorCGopt] {data/2Dexample1-CG-opt-exact-0-maxIt-1000-tolerance-1e-10-toleranceRes-0.01.txt};
	\addlegendentry{CG Optimal (inexact)}
	
	% CG Lyapunov
	\addplot+ [Green,very thick,mark=o] table [x=refinements, y=errorCGlyap] {data/2Dexample1-CG-lyap-exact-1-maxIt-1000-tolerance-1e-10-toleranceRes-0.01.txt};
	\addlegendentry{CG Lyapunov}
	\addplot+ [Green,dotted,very thick,mark=o] table [x=refinements, y=errorCGlyap] {data/2Dexample1-CG-lyap-exact-0-maxIt-1000-tolerance-1e-10-toleranceRes-0.01.txt};
	\addlegendentry{CG Lyapunov (inexact)}
	
	% Galerkin
	\addplot+ [red,very thick,mark=*] table [x=refinements, y=errorGalerkin] {data/2Dexample1-Galerkin-1-maxIt-1000-tolerance-1e-10-toleranceRes-0.01.txt};
	\addlegendentry{Galerkin}

	% Timestepping
	\addplot+ [black,very thick,mark=diamond] table [x=refinements, y=l2error] {data/2Dexample1-timestepping.txt};
	\addlegendentry{Timestepping}
	
	\end{semilogyaxis}
\end{tikzpicture}
\begin{tikzpicture}[xscale=.8,yscale=.8]
	\begin{semilogyaxis}[title={2D Example 1: Walltimes},
			xlabel={Number of refinements},
			ylabel={Wall time $[s]$},
			grid=major,
%			legend style={at={(0,1)},xshift=0.2cm,yshift=-0.2cm,anchor=north west,nodes=right			}
			]	
	% CG Optimal
	\addplot+ [blue,very thick,mark=x] table [x=refinements, y=timeCGopt] {data/2Dexample1-CG-opt-exact-1-maxIt-1000-tolerance-1e-10-toleranceRes-0.01.txt};
%	\addlegendentry{CG Optimal}
	\addplot+ [blue,dotted,very thick,mark=x] table [x=refinements, y=timeCGopt] {data/2Dexample1-CG-opt-exact-0-maxIt-1000-tolerance-1e-10-toleranceRes-0.01.txt};
%	\addlegendentry{CG Optimal (inexact)}

	% CG Lyapunov
	\addplot+ [Green,very thick,mark=o] table [x=refinements, y=timeCGlyap] {data/2Dexample1-CG-lyap-exact-1-maxIt-1000-tolerance-1e-10-toleranceRes-0.01.txt};
%	\addlegendentry{CG Lyapunov}
	\addplot+ [Green,dotted,very thick,mark=o] table [x=refinements, y=timeCGlyap] {data/2Dexample1-CG-lyap-exact-0-maxIt-1000-tolerance-1e-10-toleranceRes-0.01.txt};
%	\addlegendentry{CG Lyapunov (inexact)}

	% Galerkin
	\addplot+ [red,very thick,mark=*] table [x=refinements, y=timeGalerkin] {data/2Dexample1-Galerkin-1-maxIt-1000-tolerance-1e-10-toleranceRes-0.01.txt};
%	\addlegendentry{Galerkin}

	% Timestepping
	\addplot+ [black,very thick,mark=diamond] table [x=refinements, y=times] {data/2Dexample1-Timestepping.txt};
%	\addlegendentry{Timestepping}
	\end{semilogyaxis}
\end{tikzpicture}


% Load the data
\pgfplotstableread{data/2Dexample1-CG-opt-exact-0-maxIt-1000-tolerance-1e-10-toleranceRes-0.01.txt}\TwoDOneCGOptFalse
\pgfplotstableread{data/2Dexample1-CG-opt-exact-1-maxIt-1000-tolerance-1e-10-toleranceRes-0.01.txt}\TwoDOneCGOptTrue

\pgfplotstableread{data/2Dexample1-CG-lyap-exact-0-maxIt-1000-tolerance-1e-10-toleranceRes-0.01.txt}\TwoDOneCGLyapFalse
\pgfplotstableread{data/2Dexample1-CG-lyap-exact-1-maxIt-1000-tolerance-1e-10-toleranceRes-0.01.txt}\TwoDOneCGLyapTrue


\pgfplotstableread{data/2Dexample1-Galerkin-1-maxIt-1000-tolerance-1e-10-toleranceRes-0.01.txt}\TwoDOneGalerkin


\pgfplotstableread{data/2Dexample1-Timestepping.txt}\TwoDOneTimestepping



\pgfplotstableread{data/2Drefinements.txt}\TwoDOne


\pgfplotstablecreatecol[copy column from table={\TwoDOneCGOptTrue}{[index] 3}] {errorCGOPTexact} {\TwoDOne}
\pgfplotstablecreatecol[copy column from table={\TwoDOneCGOptTrue}{[index] 2}] {timeCGOPTexact} {\TwoDOne}

\pgfplotstablecreatecol[copy column from table={\TwoDOneCGOptFalse}{[index] 3}] {errorCGOPTinexact} {\TwoDOne}
\pgfplotstablecreatecol[copy column from table={\TwoDOneCGOptFalse}{[index] 2}] {timeCGOPTinexact} {\TwoDOne}

\pgfplotstablecreatecol[copy column from table={\TwoDOneCGLyapTrue}{[index] 3}] {errorCGLYAPexact} {\TwoDOne}
\pgfplotstablecreatecol[copy column from table={\TwoDOneCGLyapTrue}{[index] 2}] {timeCGLYAPexact} {\TwoDOne}


\pgfplotstablecreatecol[copy column from table={\TwoDOneCGLyapFalse}{[index] 3}] {errorCGLYAPinexact} {\TwoDOne}
\pgfplotstablecreatecol[copy column from table={\TwoDOneCGLyapFalse}{[index] 2}] {timeCGLYAPinexact} {\TwoDOne}


\pgfplotstablecreatecol[copy column from table={\TwoDOneGalerkin}{[index] 3}] {errorGalerkin} {\TwoDOne}
\pgfplotstablecreatecol[copy column from table={\TwoDOneGalerkin}{[index] 2}] {timeGalerkin} {\TwoDOne}

\pgfplotstablecreatecol[copy column from table={\TwoDOneTimestepping}{[index] 2}] {errorTimestepping} {\TwoDOne}
\pgfplotstablecreatecol[copy column from table={\TwoDOneTimestepping}{[index] 1}] {timeTimestepping} {\TwoDOne}




\hspace{-1cm}

\pgfplotstabletypeset[
%Rename the columns
columns/Refinements/.style={column name={Ref.}},
columns/errorCGOPTexact/.style={column name={$L_2$ error}},
columns/timeCGOPTexact/.style={column name={Time}},
columns/errorCGOPTinexact/.style={column name={$L_2$ error}},
columns/timeCGOPTinexact/.style={column name={Time}},
columns/errorCGLYAPexact/.style={column name={$L_2$ error}},
columns/timeCGLYAPexact/.style={column name={Time}},
columns/errorCGLYAPinexact/.style={column name={$L_2$ error}},
columns/timeCGLYAPinexact/.style={column name={Time}},
columns/errorGalerkin/.style={column name={$L_2$ error}},
columns/timeGalerkin/.style={column name={Time}},
columns/errorTimestepping/.style={column name={$L_2$ error}},
columns/timeTimestepping/.style={column name={Time},column type/.add={}{|}},
precision=1,
% Head row		
every head row/.style={
	before row={
		\hline
		 & \multicolumn{2}{c|}{Unknowns} & \multicolumn{2}{c|}{CG opt (exact)}
		  & \multicolumn{2}{c|}{CG opt (inexact)} & \multicolumn{2}{c|}{CG lyap (exact)} & \multicolumn{2}{c|}{CG lyap (inexact)} & \multicolumn{2}{c|}{Galerkin} & \multicolumn{2}{c|}{Timestepping}\\
	},
	after row=\hline,		
},
column type/.add={|}{},
every last row/.style={after row=\hline}	
]{\TwoDOne}

\subsection*{Example 2 ($u \in C^0$)}
\begin{tikzpicture}[xscale=.8,yscale=.8]
	\begin{semilogyaxis}[title={2D Example 2: $L_2$ convergence},
			xlabel={Number of refinements},
			ylabel={$L_2$ error},
			grid=major,
			ymax=1e0,
			legend style={at={(1,1)},xshift=2cm,yshift=-0.2cm,anchor=north east,nodes=right}
			]	
	% CG Optimal
	\addplot+ [blue,very thick,mark=x] table [x=refinements, y=errorCGopt] {data/2Dexample2-CG-opt-exact-1-maxIt-1000-tolerance-1e-10-toleranceRes-0.01.txt};
	\addlegendentry{CG Optimal}
	\addplot+ [blue,dotted,very thick,mark=x] table [x=refinements, y=errorCGopt] {data/2Dexample2-CG-opt-exact-0-maxIt-1000-tolerance-1e-10-toleranceRes-0.01.txt};
	\addlegendentry{CG Optimal (inexact)}
	
	% CG Lyapunov
	\addplot+ [Green,very thick,mark=o] table [x=refinements, y=errorCGlyap] {data/2Dexample2-CG-lyap-exact-1-maxIt-1000-tolerance-1e-10-toleranceRes-0.01.txt};
	\addlegendentry{CG Lyapunov}
	\addplot+ [Green,dotted,very thick,mark=o] table [x=refinements, y=errorCGlyap] {data/2Dexample2-CG-lyap-exact-0-maxIt-1000-tolerance-1e-10-toleranceRes-0.01.txt};
	\addlegendentry{CG Lyapunov (inexact)}
	
	% Galerkin
	\addplot+ [red,very thick,mark=*] table [x=refinements, y=errorGalerkin] {data/2Dexample2-Galerkin-1-maxIt-1000-tolerance-1e-10-toleranceRes-0.01.txt};
	\addlegendentry{Galerkin}

	% Timestepping
	\addplot+ [black,very thick,mark=diamond] table [x=refinements, y=l2error] {data/2Dexample2-timestepping.txt};
	\addlegendentry{Timestepping}
	
	\end{semilogyaxis}
\end{tikzpicture}
\begin{tikzpicture}[xscale=.8,yscale=.8]
	\begin{semilogyaxis}[title={2D Example 2: Walltimes},
			xlabel={Number of refinements},
			ylabel={Wall time $[s]$},
			grid=major,
%			legend style={at={(0,1)},xshift=0.2cm,yshift=-0.2cm,anchor=north west,nodes=right			}
			]	
	% CG Optimal
	\addplot+ [blue,very thick,mark=x] table [x=refinements, y=timeCGopt] {data/2Dexample2-CG-opt-exact-1-maxIt-1000-tolerance-1e-10-toleranceRes-0.01.txt};
%	\addlegendentry{CG Optimal}
	\addplot+ [blue,dotted,very thick,mark=x] table [x=refinements, y=timeCGopt] {data/2Dexample2-CG-opt-exact-0-maxIt-1000-tolerance-1e-10-toleranceRes-0.01.txt};
%	\addlegendentry{CG Optimal (inexact)}

	% CG Lyapunov
	\addplot+ [Green,very thick,mark=o] table [x=refinements, y=timeCGlyap] {data/2Dexample2-CG-lyap-exact-1-maxIt-1000-tolerance-1e-10-toleranceRes-0.01.txt};
%	\addlegendentry{CG Lyapunov}
	\addplot+ [Green,dotted,very thick,mark=o] table [x=refinements, y=timeCGlyap] {data/2Dexample2-CG-lyap-exact-0-maxIt-1000-tolerance-1e-10-toleranceRes-0.01.txt};
%	\addlegendentry{CG Lyapunov (inexact)}

	% Galerkin
	\addplot+ [red,very thick,mark=*] table [x=refinements, y=timeGalerkin] {data/2Dexample2-Galerkin-1-maxIt-1000-tolerance-1e-10-toleranceRes-0.01.txt};
%	\addlegendentry{Galerkin}

	% Timestepping
	\addplot+ [black,very thick,mark=diamond] table [x=refinements, y=times] {data/2Dexample2-Timestepping.txt};
%	\addlegendentry{Timestepping}
	\end{semilogyaxis}
\end{tikzpicture}


% Load the data
\pgfplotstableread{data/2Dexample2-CG-opt-exact-0-maxIt-1000-tolerance-1e-10-toleranceRes-0.01.txt}\TwoDTwoCGOptFalse
\pgfplotstableread{data/2Dexample2-CG-opt-exact-1-maxIt-1000-tolerance-1e-10-toleranceRes-0.01.txt}\TwoDTwoCGOptTrue

\pgfplotstableread{data/2Dexample2-CG-lyap-exact-0-maxIt-1000-tolerance-1e-10-toleranceRes-0.01.txt}\TwoDTwoCGLyapFalse
\pgfplotstableread{data/2Dexample2-CG-lyap-exact-1-maxIt-1000-tolerance-1e-10-toleranceRes-0.01.txt}\TwoDTwoCGLyapTrue


\pgfplotstableread{data/2Dexample2-Galerkin-1-maxIt-1000-tolerance-1e-10-toleranceRes-0.01.txt}\TwoDTwoGalerkin


\pgfplotstableread{data/2Dexample2-Timestepping.txt}\TwoDTwoTimestepping

%\pgfplotstablenew[copy column from table={\dataA}{[index] 0}]{8}{\dataC}

%\pgfplotstablenew[
%  create on use/Refinements/.style={create col/set list={1,...,8}},
%  columns={x}
%]{8}\dataC

\pgfplotstableread{data/2Drefinements.txt}\TwoDTwo


\pgfplotstablecreatecol[copy column from table={\TwoDTwoCGOptTrue}{[index] 3}] {errorCGOPTexact} {\TwoDTwo}
\pgfplotstablecreatecol[copy column from table={\TwoDTwoCGOptTrue}{[index] 2}] {timeCGOPTexact} {\TwoDTwo}

\pgfplotstablecreatecol[copy column from table={\TwoDTwoCGOptFalse}{[index] 3}] {errorCGOPTinexact} {\TwoDTwo}
\pgfplotstablecreatecol[copy column from table={\TwoDTwoCGOptFalse}{[index] 2}] {timeCGOPTinexact} {\TwoDTwo}

\pgfplotstablecreatecol[copy column from table={\TwoDTwoCGLyapTrue}{[index] 3}] {errorCGLYAPexact} {\TwoDTwo}
\pgfplotstablecreatecol[copy column from table={\TwoDTwoCGLyapTrue}{[index] 2}] {timeCGLYAPexact} {\TwoDTwo}


\pgfplotstablecreatecol[copy column from table={\TwoDTwoCGLyapFalse}{[index] 3}] {errorCGLYAPinexact} {\TwoDTwo}
\pgfplotstablecreatecol[copy column from table={\TwoDTwoCGLyapFalse}{[index] 2}] {timeCGLYAPinexact} {\TwoDTwo}


\pgfplotstablecreatecol[copy column from table={\TwoDTwoGalerkin}{[index] 3}] {errorGalerkin} {\TwoDTwo}
\pgfplotstablecreatecol[copy column from table={\TwoDTwoGalerkin}{[index] 2}] {timeGalerkin} {\TwoDTwo}

\pgfplotstablecreatecol[copy column from table={\TwoDTwoTimestepping}{[index] 2}] {errorTimestepping} {\TwoDTwo}
\pgfplotstablecreatecol[copy column from table={\TwoDTwoTimestepping}{[index] 1}] {timeTimestepping} {\TwoDTwo}




\hspace{-1cm}
\scriptsize
\pgfplotstabletypeset[
%Rename the columns
columns/Refinements/.style={column name={Ref.}},
columns/errorCGOPTexact/.style={column name={$L_2$ error}},
columns/timeCGOPTexact/.style={column name={Time}},
columns/errorCGOPTinexact/.style={column name={$L_2$ error}},
columns/timeCGOPTinexact/.style={column name={Time}},
columns/errorCGLYAPexact/.style={column name={$L_2$ error}},
columns/timeCGLYAPexact/.style={column name={Time}},
columns/errorCGLYAPinexact/.style={column name={$L_2$ error}},
columns/timeCGLYAPinexact/.style={column name={Time}},
columns/errorGalerkin/.style={column name={$L_2$ error}},
columns/timeGalerkin/.style={column name={Time}},
columns/errorTimestepping/.style={column name={$L_2$ error}},
columns/timeTimestepping/.style={column name={Time},column type/.add={}{|}},
precision=1,
% Head row		
every head row/.style={
	before row={
		\hline
		 & \multicolumn{2}{c|}{Unknowns} & \multicolumn{2}{c|}{CG opt (exact)}
		  & \multicolumn{2}{c|}{CG opt (inexact)} & \multicolumn{2}{c|}{CG lyap (exact)} & \multicolumn{2}{c|}{CG lyap (inexact)} & \multicolumn{2}{c|}{Galerkin} & \multicolumn{2}{c|}{Timestepping}\\
	},
	after row=\hline,		
},
column type/.add={|}{},
every last row/.style={after row=\hline}	
]{\TwoDTwo}

\subsection*{Example 3 ($u \in L_2$)}
\begin{tikzpicture}[xscale=.8,yscale=.8]
	\begin{semilogyaxis}[title={2D Example 3: $L_2$ convergence},
			xlabel={Number of refinements},
			ylabel={$L_2$ error},
			grid=major,
			ymax=1e0,
			legend style={at={(1,1)},xshift=2cm,yshift=-0.2cm,anchor=north east,nodes=right}
			]	
	% CG Optimal
	\addplot+ [blue,very thick,mark=x] table [x=refinements, y=errorCGopt] {data/2Dexample3-CG-opt-exact-1-maxIt-1000-tolerance-1e-10-toleranceRes-0.01.txt};
	\addlegendentry{CG Optimal}
	\addplot+ [blue,dotted,very thick,mark=x] table [x=refinements, y=errorCGopt] {data/2Dexample3-CG-opt-exact-0-maxIt-1000-tolerance-1e-10-toleranceRes-0.01.txt};
	\addlegendentry{CG Optimal (inexact)}
	
	% CG Lyapunov
	\addplot+ [Green,very thick,mark=o] table [x=refinements, y=errorCGlyap] {data/2Dexample3-CG-lyap-exact-1-maxIt-1000-tolerance-1e-10-toleranceRes-0.01.txt};
	\addlegendentry{CG Lyapunov}
	\addplot+ [Green,dotted,very thick,mark=o] table [x=refinements, y=errorCGlyap] {data/2Dexample3-CG-lyap-exact-0-maxIt-1000-tolerance-1e-10-toleranceRes-0.01.txt};
	\addlegendentry{CG Lyapunov (inexact)}
	
	% Galerkin
	\addplot+ [red,very thick,mark=*] table [x=refinements, y=errorGalerkin] {data/2Dexample3-Galerkin-1-maxIt-1000-tolerance-1e-10-toleranceRes-0.01.txt};
	\addlegendentry{Galerkin}

	% Timestepping
	\addplot+ [black,very thick,mark=diamond] table [x=refinements, y=l2error] {data/2Dexample3-timestepping.txt};
	\addlegendentry{Timestepping}
	
	\end{semilogyaxis}
\end{tikzpicture}
\begin{tikzpicture}[xscale=.8,yscale=.8]
	\begin{semilogyaxis}[title={2D Example 3: Walltimes},
			xlabel={Number of refinements},
			ylabel={Wall time $[s]$},
			grid=major,
%			legend style={at={(0,1)},xshift=0.2cm,yshift=-0.2cm,anchor=north west,nodes=right			}
			]	
	% CG Optimal
	\addplot+ [blue,very thick,mark=x] table [x=refinements, y=timeCGopt] {data/2Dexample3-CG-opt-exact-1-maxIt-1000-tolerance-1e-10-toleranceRes-0.01.txt};
%	\addlegendentry{CG Optimal}
	\addplot+ [blue,dotted,very thick,mark=x] table [x=refinements, y=timeCGopt] {data/2Dexample3-CG-opt-exact-0-maxIt-1000-tolerance-1e-10-toleranceRes-0.01.txt};
%	\addlegendentry{CG Optimal (inexact)}

	% CG Lyapunov
	\addplot+ [Green,very thick,mark=o] table [x=refinements, y=timeCGlyap] {data/2Dexample3-CG-lyap-exact-1-maxIt-1000-tolerance-1e-10-toleranceRes-0.01.txt};
%	\addlegendentry{CG Lyapunov}
	\addplot+ [Green,dotted,very thick,mark=o] table [x=refinements, y=timeCGlyap] {data/2Dexample3-CG-lyap-exact-0-maxIt-1000-tolerance-1e-10-toleranceRes-0.01.txt};
%	\addlegendentry{CG Lyapunov (inexact)}

	% Galerkin
	\addplot+ [red,very thick,mark=*] table [x=refinements, y=timeGalerkin] {data/2Dexample3-Galerkin-1-maxIt-1000-tolerance-1e-10-toleranceRes-0.01.txt};
%	\addlegendentry{Galerkin}

	% Timestepping
	\addplot+ [black,very thick,mark=diamond] table [x=refinements, y=times] {data/2Dexample3-Timestepping.txt};
%	\addlegendentry{Timestepping}
	\end{semilogyaxis}
\end{tikzpicture}


% Load the data
\pgfplotstableread{data/2Dexample3-CG-opt-exact-0-maxIt-1000-tolerance-1e-10-toleranceRes-0.01.txt}\TwoDThreeCGOptFalse
\pgfplotstableread{data/2Dexample3-CG-opt-exact-1-maxIt-1000-tolerance-1e-10-toleranceRes-0.01.txt}\TwoDThreeCGOptTrue

\pgfplotstableread{data/2Dexample3-CG-lyap-exact-0-maxIt-1000-tolerance-1e-10-toleranceRes-0.01.txt}\TwoDThreeCGLyapFalse
\pgfplotstableread{data/2Dexample3-CG-lyap-exact-1-maxIt-1000-tolerance-1e-10-toleranceRes-0.01.txt}\TwoDThreeCGLyapTrue


\pgfplotstableread{data/2Dexample3-Galerkin-1-maxIt-1000-tolerance-1e-10-toleranceRes-0.01.txt}\TwoDThreeGalerkin


\pgfplotstableread{data/2Dexample3-Timestepping.txt}\TwoDThreeTimestepping

%\pgfplotstablenew[copy column from table={\dataA}{[index] 0}]{8}{\dataC}

%\pgfplotstablenew[
%  create on use/Refinements/.style={create col/set list={1,...,8}},
%  columns={x}
%]{8}\dataC

\pgfplotstableread{data/2Drefinements.txt}\TwoDThree


\pgfplotstablecreatecol[copy column from table={\TwoDThreeCGOptTrue}{[index] 3}] {errorCGOPTexact} {\TwoDThree}
\pgfplotstablecreatecol[copy column from table={\TwoDThreeCGOptTrue}{[index] 2}] {timeCGOPTexact} {\TwoDThree}

\pgfplotstablecreatecol[copy column from table={\TwoDThreeCGOptFalse}{[index] 3}] {errorCGOPTinexact} {\TwoDThree}
\pgfplotstablecreatecol[copy column from table={\TwoDThreeCGOptFalse}{[index] 2}] {timeCGOPTinexact} {\TwoDThree}

\pgfplotstablecreatecol[copy column from table={\TwoDThreeCGLyapTrue}{[index] 3}] {errorCGLYAPexact} {\TwoDThree}
\pgfplotstablecreatecol[copy column from table={\TwoDThreeCGLyapTrue}{[index] 2}] {timeCGLYAPexact} {\TwoDThree}


\pgfplotstablecreatecol[copy column from table={\TwoDThreeCGLyapFalse}{[index] 3}] {errorCGLYAPinexact} {\TwoDThree}
\pgfplotstablecreatecol[copy column from table={\TwoDThreeCGLyapFalse}{[index] 2}] {timeCGLYAPinexact} {\TwoDThree}


\pgfplotstablecreatecol[copy column from table={\TwoDThreeGalerkin}{[index] 3}] {errorGalerkin} {\TwoDThree}
\pgfplotstablecreatecol[copy column from table={\TwoDThreeGalerkin}{[index] 2}] {timeGalerkin} {\TwoDThree}

\pgfplotstablecreatecol[copy column from table={\TwoDThreeTimestepping}{[index] 2}] {errorTimestepping} {\TwoDThree}
\pgfplotstablecreatecol[copy column from table={\TwoDThreeTimestepping}{[index] 1}] {timeTimestepping} {\TwoDThree}




\hspace{-1cm}
\scriptsize
\pgfplotstabletypeset[
%Rename the columns
columns/Refinements/.style={column name={Ref.}},
columns/errorCGOPTexact/.style={column name={$L_2$ error}},
columns/timeCGOPTexact/.style={column name={Time}},
columns/errorCGOPTinexact/.style={column name={$L_2$ error}},
columns/timeCGOPTinexact/.style={column name={Time}},
columns/errorCGLYAPexact/.style={column name={$L_2$ error}},
columns/timeCGLYAPexact/.style={column name={Time}},
columns/errorCGLYAPinexact/.style={column name={$L_2$ error}},
columns/timeCGLYAPinexact/.style={column name={Time}},
columns/errorGalerkin/.style={column name={$L_2$ error}},
columns/timeGalerkin/.style={column name={Time}},
columns/errorTimestepping/.style={column name={$L_2$ error}},
columns/timeTimestepping/.style={column name={Time},column type/.add={}{|}},
precision=1,
% Head row		
every head row/.style={
	before row={
		\hline
		 & \multicolumn{2}{c|}{Unknowns} & \multicolumn{2}{c|}{CG opt (exact)}
		  & \multicolumn{2}{c|}{CG opt (inexact)} & \multicolumn{2}{c|}{CG lyap (exact)} & \multicolumn{2}{c|}{CG lyap (inexact)} & \multicolumn{2}{c|}{Galerkin} & \multicolumn{2}{c|}{Timestepping}\\
	},
	after row=\hline,		
},
column type/.add={|}{},
every last row/.style={after row=\hline}	
]{\TwoDThree}

\newpage
\section*{3D}
\subsection*{Example 1 ($u \in C^\infty$)}
\begin{tikzpicture}[xscale=.8,yscale=.8]
	\begin{semilogyaxis}[title={3D Example 1: $L_2$ convergence},
			xlabel={Number of refinements},
			ylabel={$L_2$ error},
			grid=major,
			ymax=1e2,
			legend style={at={(1,1)},xshift=2cm,yshift=-0.2cm,anchor=north east,nodes=right}
			]	
	% CG Optimal
	\addplot+ [blue,very thick,mark=x] table [x=refinements, y=errorCGopt] {data/3Dexample1-CG-opt-exact-1-maxIt-1000-tolerance-1e-10-toleranceRes-0.01.txt};
	\addlegendentry{CG Optimal}
	\addplot+ [blue,dotted,very thick,mark=x] table [x=refinements, y=errorCGopt] {data/3Dexample1-CG-opt-exact-0-maxIt-1000-tolerance-1e-10-toleranceRes-0.01.txt};
	\addlegendentry{CG Optimal (inexact)}
	
	% CG Lyapunov
	\addplot+ [Green,very thick,mark=o] table [x=refinements, y=errorCGlyap] {data/3Dexample1-CG-lyap-exact-1-maxIt-1000-tolerance-1e-10-toleranceRes-0.01.txt};
	\addlegendentry{CG Lyapunov}
	\addplot+ [Green,dotted,very thick,mark=o] table [x=refinements, y=errorCGlyap] {data/3Dexample1-CG-lyap-exact-0-maxIt-1000-tolerance-1e-10-toleranceRes-0.01.txt};
	\addlegendentry{CG Lyapunov (inexact)}
	
	% Galerkin
	\addplot+ [red,very thick,mark=*] table [x=refinements, y=errorGalerkin] {data/3Dexample1-Galerkin-1-maxIt-1000-tolerance-1e-10-toleranceRes-0.01.txt};
	\addlegendentry{Galerkin}

	% Timestepping
	\addplot+ [black,very thick,mark=diamond] table [x=refinements, y=l2error] {data/3Dexample1-timestepping.txt};
	\addlegendentry{Timestepping}
	
	\end{semilogyaxis}
\end{tikzpicture}
\begin{tikzpicture}[xscale=.8,yscale=.8]
	\begin{semilogyaxis}[title={3D Example 1: Walltimes},
			xlabel={Number of refinements},
			ylabel={Wall time $[s]$},
			grid=major,
%			legend style={at={(0,1)},xshift=0.2cm,yshift=-0.2cm,anchor=north west,nodes=right			}
			]	
	% CG Optimal
	\addplot+ [blue,very thick,mark=x] table [x=refinements, y=timeCGopt] {data/3Dexample1-CG-opt-exact-1-maxIt-1000-tolerance-1e-10-toleranceRes-0.01.txt};
%	\addlegendentry{CG Optimal}
	\addplot+ [blue,dotted,very thick,mark=x] table [x=refinements, y=timeCGopt] {data/3Dexample1-CG-opt-exact-0-maxIt-1000-tolerance-1e-10-toleranceRes-0.01.txt};
%	\addlegendentry{CG Optimal (inexact)}

	% CG Lyapunov
	\addplot+ [Green,very thick,mark=o] table [x=refinements, y=timeCGlyap] {data/3Dexample1-CG-lyap-exact-1-maxIt-1000-tolerance-1e-10-toleranceRes-0.01.txt};
%	\addlegendentry{CG Lyapunov}
	\addplot+ [Green,dotted,very thick,mark=o] table [x=refinements, y=timeCGlyap] {data/3Dexample1-CG-lyap-exact-0-maxIt-1000-tolerance-1e-10-toleranceRes-0.01.txt};
%	\addlegendentry{CG Lyapunov (inexact)}

	% Galerkin
	\addplot+ [red,very thick,mark=*] table [x=refinements, y=timeGalerkin] {data/3Dexample1-Galerkin-1-maxIt-1000-tolerance-1e-10-toleranceRes-0.01.txt};
%	\addlegendentry{Galerkin}

	% Timestepping
	\addplot+ [black,very thick,mark=diamond] table [x=refinements, y=times] {data/3Dexample1-Timestepping.txt};
%	\addlegendentry{Timestepping}
	\end{semilogyaxis}
\end{tikzpicture}


% Load the data
\pgfplotstableread{data/3Dexample1-CG-opt-exact-0-maxIt-1000-tolerance-1e-10-toleranceRes-0.01.txt}\ThreeDOneCGOptFalse
\pgfplotstableread{data/3Dexample1-CG-opt-exact-1-maxIt-1000-tolerance-1e-10-toleranceRes-0.01.txt}\ThreeDOneCGOptTrue

\pgfplotstableread{data/3Dexample1-CG-lyap-exact-0-maxIt-1000-tolerance-1e-10-toleranceRes-0.01.txt}\ThreeDOneCGLyapFalse
\pgfplotstableread{data/3Dexample1-CG-lyap-exact-1-maxIt-1000-tolerance-1e-10-toleranceRes-0.01.txt}\ThreeDOneCGLyapTrue


\pgfplotstableread{data/3Dexample1-Galerkin-1-maxIt-1000-tolerance-1e-10-toleranceRes-0.01.txt}\ThreeDOneGalerkin


\pgfplotstableread{data/3Dexample1-Timestepping.txt}\ThreeDOneTimestepping



\pgfplotstableread{data/3Drefinements.txt}\ThreeDOne


\pgfplotstablecreatecol[copy column from table={\ThreeDOneCGOptTrue}{[index] 3}] {errorCGOPTexact} {\ThreeDOne}
\pgfplotstablecreatecol[copy column from table={\ThreeDOneCGOptTrue}{[index] 2}] {timeCGOPTexact} {\ThreeDOne}

\pgfplotstablecreatecol[copy column from table={\ThreeDOneCGOptFalse}{[index] 3}] {errorCGOPTinexact} {\ThreeDOne}
\pgfplotstablecreatecol[copy column from table={\ThreeDOneCGOptFalse}{[index] 2}] {timeCGOPTinexact} {\ThreeDOne}

\pgfplotstablecreatecol[copy column from table={\ThreeDOneCGLyapTrue}{[index] 3}] {errorCGLYAPexact} {\ThreeDOne}
\pgfplotstablecreatecol[copy column from table={\ThreeDOneCGLyapTrue}{[index] 2}] {timeCGLYAPexact} {\ThreeDOne}


\pgfplotstablecreatecol[copy column from table={\ThreeDOneCGLyapFalse}{[index] 3}] {errorCGLYAPinexact} {\ThreeDOne}
\pgfplotstablecreatecol[copy column from table={\ThreeDOneCGLyapFalse}{[index] 2}] {timeCGLYAPinexact} {\ThreeDOne}


\pgfplotstablecreatecol[copy column from table={\ThreeDOneGalerkin}{[index] 3}] {errorGalerkin} {\ThreeDOne}
\pgfplotstablecreatecol[copy column from table={\ThreeDOneGalerkin}{[index] 2}] {timeGalerkin} {\ThreeDOne}

\pgfplotstablecreatecol[copy column from table={\ThreeDOneTimestepping}{[index] 2}] {errorTimestepping} {\ThreeDOne}
\pgfplotstablecreatecol[copy column from table={\ThreeDOneTimestepping}{[index] 1}] {timeTimestepping} {\ThreeDOne}




\hspace{-1cm}

\pgfplotstabletypeset[
%Rename the columns
columns/Refinements/.style={column name={Ref.}},
columns/errorCGOPTexact/.style={column name={$L_2$ error}},
columns/timeCGOPTexact/.style={column name={Time}},
columns/errorCGOPTinexact/.style={column name={$L_2$ error}},
columns/timeCGOPTinexact/.style={column name={Time}},
columns/errorCGLYAPexact/.style={column name={$L_2$ error}},
columns/timeCGLYAPexact/.style={column name={Time}},
columns/errorCGLYAPinexact/.style={column name={$L_2$ error}},
columns/timeCGLYAPinexact/.style={column name={Time}},
columns/errorGalerkin/.style={column name={$L_2$ error}},
columns/timeGalerkin/.style={column name={Time}},
columns/errorTimestepping/.style={column name={$L_2$ error}},
columns/timeTimestepping/.style={column name={Time},column type/.add={}{|}},
precision=1,
% Head row		
every head row/.style={
	before row={
		\hline
		 & \multicolumn{2}{c|}{Unknowns} & \multicolumn{2}{c|}{CG opt (exact)}
		  & \multicolumn{2}{c|}{CG opt (inexact)} & \multicolumn{2}{c|}{CG lyap (exact)} & \multicolumn{2}{c|}{CG lyap (inexact)} & \multicolumn{2}{c|}{Galerkin} & \multicolumn{2}{c|}{Timestepping}\\
	},
	after row=\hline,		
},
column type/.add={|}{},
every last row/.style={after row=\hline}	
]{\ThreeDOne}

\subsection*{Example 2 ($u \in C^0$)}
\begin{tikzpicture}[xscale=.8,yscale=.8]
	\begin{semilogyaxis}[title={3D Example 2: $L_2$ convergence},
			xlabel={Number of refinements},
			ylabel={$L_2$ error},
			grid=major,
			ymax=1e0,
			legend style={at={(1,1)},xshift=2cm,yshift=-0.2cm,anchor=north east,nodes=right}
			]	
	% CG Optimal
	\addplot+ [blue,very thick,mark=x] table [x=refinements, y=errorCGopt] {data/3Dexample2-CG-opt-exact-1-maxIt-1000-tolerance-1e-10-toleranceRes-0.01.txt};
	\addlegendentry{CG Optimal}
	\addplot+ [blue,dotted,very thick,mark=x] table [x=refinements, y=errorCGopt] {data/3Dexample2-CG-opt-exact-0-maxIt-1000-tolerance-1e-10-toleranceRes-0.01.txt};
	\addlegendentry{CG Optimal (inexact)}
	
	% CG Lyapunov
	\addplot+ [Green,very thick,mark=o] table [x=refinements, y=errorCGlyap] {data/3Dexample2-CG-lyap-exact-1-maxIt-1000-tolerance-1e-10-toleranceRes-0.01.txt};
	\addlegendentry{CG Lyapunov}
	\addplot+ [Green,dotted,very thick,mark=o] table [x=refinements, y=errorCGlyap] {data/3Dexample2-CG-lyap-exact-0-maxIt-1000-tolerance-1e-10-toleranceRes-0.01.txt};
	\addlegendentry{CG Lyapunov (inexact)}
	
	% Galerkin
	\addplot+ [red,very thick,mark=*] table [x=refinements, y=errorGalerkin] {data/3Dexample2-Galerkin-1-maxIt-1000-tolerance-1e-10-toleranceRes-0.01.txt};
	\addlegendentry{Galerkin}

	% Timestepping
	\addplot+ [black,very thick,mark=diamond] table [x=refinements, y=l2error] {data/3Dexample2-timestepping.txt};
	\addlegendentry{Timestepping}
	
	\end{semilogyaxis}
\end{tikzpicture}
\begin{tikzpicture}[xscale=.8,yscale=.8]
	\begin{semilogyaxis}[title={3D Example 2: Walltimes},
			xlabel={Number of refinements},
			ylabel={Wall time $[s]$},
			grid=major,
%			legend style={at={(0,1)},xshift=0.2cm,yshift=-0.2cm,anchor=north west,nodes=right			}
			]	
	% CG Optimal
	\addplot+ [blue,very thick,mark=x] table [x=refinements, y=timeCGopt] {data/3Dexample2-CG-opt-exact-1-maxIt-1000-tolerance-1e-10-toleranceRes-0.01.txt};
%	\addlegendentry{CG Optimal}
	\addplot+ [blue,dotted,very thick,mark=x] table [x=refinements, y=timeCGopt] {data/3Dexample2-CG-opt-exact-0-maxIt-1000-tolerance-1e-10-toleranceRes-0.01.txt};
%	\addlegendentry{CG Optimal (inexact)}

	% CG Lyapunov
	\addplot+ [Green,very thick,mark=o] table [x=refinements, y=timeCGlyap] {data/3Dexample2-CG-lyap-exact-1-maxIt-1000-tolerance-1e-10-toleranceRes-0.01.txt};
%	\addlegendentry{CG Lyapunov}
	\addplot+ [Green,dotted,very thick,mark=o] table [x=refinements, y=timeCGlyap] {data/3Dexample2-CG-lyap-exact-0-maxIt-1000-tolerance-1e-10-toleranceRes-0.01.txt};
%	\addlegendentry{CG Lyapunov (inexact)}

	% Galerkin
	\addplot+ [red,very thick,mark=*] table [x=refinements, y=timeGalerkin] {data/3Dexample2-Galerkin-1-maxIt-1000-tolerance-1e-10-toleranceRes-0.01.txt};
%	\addlegendentry{Galerkin}

	% Timestepping
	\addplot+ [black,very thick,mark=diamond] table [x=refinements, y=times] {data/3Dexample2-Timestepping.txt};
%	\addlegendentry{Timestepping}
	\end{semilogyaxis}
\end{tikzpicture}


% Load the data
\pgfplotstableread{data/3Dexample2-CG-opt-exact-0-maxIt-1000-tolerance-1e-10-toleranceRes-0.01.txt}\ThreeDTwoCGOptFalse
\pgfplotstableread{data/3Dexample2-CG-opt-exact-1-maxIt-1000-tolerance-1e-10-toleranceRes-0.01.txt}\ThreeDTwoCGOptTrue

\pgfplotstableread{data/3Dexample2-CG-lyap-exact-0-maxIt-1000-tolerance-1e-10-toleranceRes-0.01.txt}\ThreeDTwoCGLyapFalse
\pgfplotstableread{data/3Dexample2-CG-lyap-exact-1-maxIt-1000-tolerance-1e-10-toleranceRes-0.01.txt}\ThreeDTwoCGLyapTrue


\pgfplotstableread{data/3Dexample2-Galerkin-1-maxIt-1000-tolerance-1e-10-toleranceRes-0.01.txt}\ThreeDTwoGalerkin


\pgfplotstableread{data/3Dexample2-Timestepping.txt}\ThreeDTwoTimestepping



\pgfplotstableread{data/3Drefinements.txt}\ThreeDTwo


\pgfplotstablecreatecol[copy column from table={\ThreeDTwoCGOptTrue}{[index] 3}] {errorCGOPTexact} {\ThreeDTwo}
\pgfplotstablecreatecol[copy column from table={\ThreeDTwoCGOptTrue}{[index] 2}] {timeCGOPTexact} {\ThreeDTwo}

\pgfplotstablecreatecol[copy column from table={\ThreeDTwoCGOptFalse}{[index] 3}] {errorCGOPTinexact} {\ThreeDTwo}
\pgfplotstablecreatecol[copy column from table={\ThreeDTwoCGOptFalse}{[index] 2}] {timeCGOPTinexact} {\ThreeDTwo}

\pgfplotstablecreatecol[copy column from table={\ThreeDTwoCGLyapTrue}{[index] 3}] {errorCGLYAPexact} {\ThreeDTwo}
\pgfplotstablecreatecol[copy column from table={\ThreeDTwoCGLyapTrue}{[index] 2}] {timeCGLYAPexact} {\ThreeDTwo}


\pgfplotstablecreatecol[copy column from table={\ThreeDTwoCGLyapFalse}{[index] 3}] {errorCGLYAPinexact} {\ThreeDTwo}
\pgfplotstablecreatecol[copy column from table={\ThreeDTwoCGLyapFalse}{[index] 2}] {timeCGLYAPinexact} {\ThreeDTwo}


\pgfplotstablecreatecol[copy column from table={\ThreeDTwoGalerkin}{[index] 3}] {errorGalerkin} {\ThreeDTwo}
\pgfplotstablecreatecol[copy column from table={\ThreeDTwoGalerkin}{[index] 2}] {timeGalerkin} {\ThreeDTwo}

\pgfplotstablecreatecol[copy column from table={\ThreeDTwoTimestepping}{[index] 2}] {errorTimestepping} {\ThreeDTwo}
\pgfplotstablecreatecol[copy column from table={\ThreeDTwoTimestepping}{[index] 1}] {timeTimestepping} {\ThreeDTwo}




\hspace{-1cm}

\pgfplotstabletypeset[
%Rename the columns
columns/Refinements/.style={column name={Ref.}},
columns/errorCGOPTexact/.style={column name={$L_2$ error}},
columns/timeCGOPTexact/.style={column name={Time}},
columns/errorCGOPTinexact/.style={column name={$L_2$ error}},
columns/timeCGOPTinexact/.style={column name={Time}},
columns/errorCGLYAPexact/.style={column name={$L_2$ error}},
columns/timeCGLYAPexact/.style={column name={Time}},
columns/errorCGLYAPinexact/.style={column name={$L_2$ error}},
columns/timeCGLYAPinexact/.style={column name={Time}},
columns/errorGalerkin/.style={column name={$L_2$ error}},
columns/timeGalerkin/.style={column name={Time}},
columns/errorTimestepping/.style={column name={$L_2$ error}},
columns/timeTimestepping/.style={column name={Time},column type/.add={}{|}},
precision=1,
% Head row		
every head row/.style={
	before row={
		\hline
		 & \multicolumn{2}{c|}{Unknowns} & \multicolumn{2}{c|}{CG opt (exact)}
		  & \multicolumn{2}{c|}{CG opt (inexact)} & \multicolumn{2}{c|}{CG lyap (exact)} & \multicolumn{2}{c|}{CG lyap (inexact)} & \multicolumn{2}{c|}{Galerkin} & \multicolumn{2}{c|}{Timestepping}\\
	},
	after row=\hline,		
},
column type/.add={|}{},
every last row/.style={after row=\hline}	
]{\ThreeDTwo}

\subsection*{Example 3 ($u \in L_2$)}
\begin{tikzpicture}[xscale=.8,yscale=.8]
	\begin{semilogyaxis}[title={3D Example 3: $L_2$ convergence},
			xlabel={Number of refinements},
			ylabel={$L_2$ error},
			grid=major,
			ymax=1e0,
			legend style={at={(1,1)},xshift=2cm,yshift=-0.2cm,anchor=north east,nodes=right}
			]	
	% CG Optimal
	\addplot+ [blue,very thick,mark=x] table [x=refinements, y=errorCGopt] {data/3Dexample3-CG-opt-exact-1-maxIt-1000-tolerance-1e-10-toleranceRes-0.01.txt};
	\addlegendentry{CG Optimal}
	\addplot+ [blue,dotted,very thick,mark=x] table [x=refinements, y=errorCGopt] {data/3Dexample3-CG-opt-exact-0-maxIt-1000-tolerance-1e-10-toleranceRes-0.01.txt};
	\addlegendentry{CG Optimal (inexact)}
	
	% CG Lyapunov
	\addplot+ [Green,very thick,mark=o] table [x=refinements, y=errorCGlyap] {data/3Dexample3-CG-lyap-exact-1-maxIt-1000-tolerance-1e-10-toleranceRes-0.01.txt};
	\addlegendentry{CG Lyapunov}
	\addplot+ [Green,dotted,very thick,mark=o] table [x=refinements, y=errorCGlyap] {data/3Dexample3-CG-lyap-exact-0-maxIt-1000-tolerance-1e-10-toleranceRes-0.01.txt};
	\addlegendentry{CG Lyapunov (inexact)}
	
	% Galerkin
	\addplot+ [red,very thick,mark=*] table [x=refinements, y=errorGalerkin] {data/3Dexample3-Galerkin-1-maxIt-1000-tolerance-1e-10-toleranceRes-0.01.txt};
	\addlegendentry{Galerkin}

	% Timestepping
	\addplot+ [black,very thick,mark=diamond] table [x=refinements, y=l2error] {data/3Dexample3-timestepping.txt};
	\addlegendentry{Timestepping}
	
	\end{semilogyaxis}
\end{tikzpicture}
\begin{tikzpicture}[xscale=.8,yscale=.8]
	\begin{semilogyaxis}[title={3D Example 3: Walltimes},
			xlabel={Number of refinements},
			ylabel={Wall time $[s]$},
			grid=major,
%			legend style={at={(0,1)},xshift=0.2cm,yshift=-0.2cm,anchor=north west,nodes=right			}
			]	
	% CG Optimal
	\addplot+ [blue,very thick,mark=x] table [x=refinements, y=timeCGopt] {data/3Dexample3-CG-opt-exact-1-maxIt-1000-tolerance-1e-10-toleranceRes-0.01.txt};
%	\addlegendentry{CG Optimal}
	\addplot+ [blue,dotted,very thick,mark=x] table [x=refinements, y=timeCGopt] {data/3Dexample3-CG-opt-exact-0-maxIt-1000-tolerance-1e-10-toleranceRes-0.01.txt};
%	\addlegendentry{CG Optimal (inexact)}

	% CG Lyapunov
	\addplot+ [Green,very thick,mark=o] table [x=refinements, y=timeCGlyap] {data/3Dexample3-CG-lyap-exact-1-maxIt-1000-tolerance-1e-10-toleranceRes-0.01.txt};
%	\addlegendentry{CG Lyapunov}
	\addplot+ [Green,dotted,very thick,mark=o] table [x=refinements, y=timeCGlyap] {data/3Dexample3-CG-lyap-exact-0-maxIt-1000-tolerance-1e-10-toleranceRes-0.01.txt};
%	\addlegendentry{CG Lyapunov (inexact)}

	% Galerkin
	\addplot+ [red,very thick,mark=*] table [x=refinements, y=timeGalerkin] {data/3Dexample3-Galerkin-1-maxIt-1000-tolerance-1e-10-toleranceRes-0.01.txt};
%	\addlegendentry{Galerkin}

	% Timestepping
	\addplot+ [black,very thick,mark=diamond] table [x=refinements, y=times] {data/3Dexample3-Timestepping.txt};
%	\addlegendentry{Timestepping}
	\end{semilogyaxis}
\end{tikzpicture}


% Load the data
\pgfplotstableread{data/3Dexample3-CG-opt-exact-0-maxIt-1000-tolerance-1e-10-toleranceRes-0.01.txt}\ThreeDThreeCGOptFalse
\pgfplotstableread{data/3Dexample3-CG-opt-exact-1-maxIt-1000-tolerance-1e-10-toleranceRes-0.01.txt}\ThreeDThreeCGOptTrue

\pgfplotstableread{data/3Dexample3-CG-lyap-exact-0-maxIt-1000-tolerance-1e-10-toleranceRes-0.01.txt}\ThreeDThreeCGLyapFalse
\pgfplotstableread{data/3Dexample3-CG-lyap-exact-1-maxIt-1000-tolerance-1e-10-toleranceRes-0.01.txt}\ThreeDThreeCGLyapTrue


\pgfplotstableread{data/3Dexample3-Galerkin-1-maxIt-1000-tolerance-1e-10-toleranceRes-0.01.txt}\ThreeDThreeGalerkin


\pgfplotstableread{data/3Dexample3-Timestepping.txt}\ThreeDThreeTimestepping



\pgfplotstableread{data/3Drefinements.txt}\ThreeDThree


\pgfplotstablecreatecol[copy column from table={\ThreeDThreeCGOptTrue}{[index] 3}] {errorCGOPTexact} {\ThreeDThree}
\pgfplotstablecreatecol[copy column from table={\ThreeDThreeCGOptTrue}{[index] 2}] {timeCGOPTexact} {\ThreeDThree}

\pgfplotstablecreatecol[copy column from table={\ThreeDThreeCGOptFalse}{[index] 3}] {errorCGOPTinexact} {\ThreeDThree}
\pgfplotstablecreatecol[copy column from table={\ThreeDThreeCGOptFalse}{[index] 2}] {timeCGOPTinexact} {\ThreeDThree}

\pgfplotstablecreatecol[copy column from table={\ThreeDThreeCGLyapTrue}{[index] 3}] {errorCGLYAPexact} {\ThreeDThree}
\pgfplotstablecreatecol[copy column from table={\ThreeDThreeCGLyapTrue}{[index] 2}] {timeCGLYAPexact} {\ThreeDThree}


\pgfplotstablecreatecol[copy column from table={\ThreeDThreeCGLyapFalse}{[index] 3}] {errorCGLYAPinexact} {\ThreeDThree}
\pgfplotstablecreatecol[copy column from table={\ThreeDThreeCGLyapFalse}{[index] 2}] {timeCGLYAPinexact} {\ThreeDThree}


\pgfplotstablecreatecol[copy column from table={\ThreeDThreeGalerkin}{[index] 3}] {errorGalerkin} {\ThreeDThree}
\pgfplotstablecreatecol[copy column from table={\ThreeDThreeGalerkin}{[index] 2}] {timeGalerkin} {\ThreeDThree}

\pgfplotstablecreatecol[copy column from table={\ThreeDThreeTimestepping}{[index] 2}] {errorTimestepping} {\ThreeDThree}
\pgfplotstablecreatecol[copy column from table={\ThreeDThreeTimestepping}{[index] 1}] {timeTimestepping} {\ThreeDThree}




\hspace{-1cm}

\pgfplotstabletypeset[
%Rename the columns
columns/Refinements/.style={column name={Ref.}},
columns/errorCGOPTexact/.style={column name={$L_2$ error}},
columns/timeCGOPTexact/.style={column name={Time}},
columns/errorCGOPTinexact/.style={column name={$L_2$ error}},
columns/timeCGOPTinexact/.style={column name={Time}},
columns/errorCGLYAPexact/.style={column name={$L_2$ error}},
columns/timeCGLYAPexact/.style={column name={Time}},
columns/errorCGLYAPinexact/.style={column name={$L_2$ error}},
columns/timeCGLYAPinexact/.style={column name={Time}},
columns/errorGalerkin/.style={column name={$L_2$ error}},
columns/timeGalerkin/.style={column name={Time}},
columns/errorTimestepping/.style={column name={$L_2$ error}},
columns/timeTimestepping/.style={column name={Time},column type/.add={}{|}},
precision=1,
% Head row		
every head row/.style={
	before row={
		\hline
		 & \multicolumn{2}{c|}{Unknowns} & \multicolumn{2}{c|}{CG opt (exact)}
		  & \multicolumn{2}{c|}{CG opt (inexact)} & \multicolumn{2}{c|}{CG lyap (exact)} & \multicolumn{2}{c|}{CG lyap (inexact)} & \multicolumn{2}{c|}{Galerkin} & \multicolumn{2}{c|}{Timestepping}\\
	},
	after row=\hline,		
},
column type/.add={|}{},
every last row/.style={after row=\hline}	
]{\ThreeDThree}



\end{document}
